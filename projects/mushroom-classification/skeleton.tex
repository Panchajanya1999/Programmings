% Please make sure you insert your
% data according to the instructions in PoSauthmanual.pdf
\documentclass[a4paper,11pt]{article}
\usepackage{pos}

\title{Project on Mushroom Classification using Classifiers}
% \ShortTitle{}

\author[a]{Panchajanya Sarkar}
\author[b]{Pravat Patra}
\author[c]{Tejas Posupo}

\affiliation[*]{Central University of Rajasthan,\\
  NH-8, Ajmer, 305817, India}

\emailAdd{me@panchajanya.dev}
\emailAdd{pravatapatra1997@gmail.com}
\emailAdd{tejasposupo@gmail.com}

\abstract{...}

\FullConference{%
  Department of Data Science and Analytics\\
  21st November, 2022, 0930HRS \\
  CURAJ, Ajmer, Rajasthan, 305817
}

%% \tableofcontents

\begin{document}
    \maketitle
        \section{Abstract}
            Abstract on Mushroom Classification using Data mining.
            This paper focuses on the use of classification algorithms to classify the mushrooms into edible and poisonous. Mushrooms dataset is composed of records of different types of mushrooms, which are edible or non-edible. WEKA (Waikato Environment for Knowledge Analysis) is used for implementation of the classification techniques. The dataset is divided into training and testing sets. The training set is used to train the classifier and the testing set is used to test the classifier. The classification algorithms used are J48, Naive Bayes, Random Forest, and Logistic Regression. The results are compared and the best classifier is chosen. The best classifier is used to classify the mushrooms into edible and poisonous. The results are compared with the actual results and the accuracy is calculated. The accuracy of the best classifier is 94\%.
        \section{Literature Review}
            \textbf{Literature Review on Mushroom Analysis}

            A literature review is a critical summary of existing research on a particular topic. In the case of mushroom classification, this review will focus on existing methods for classifying mushrooms and their effectiveness.

            Mushroom classification is an important task for many reasons. Mushrooms are a diverse group of organisms that can be found in a variety of habitats. Some mushrooms are edible and can be a valuable source of nutrition, while others are poisonous and can cause serious health problems if consumed. Accurate classification is therefore crucial for both scientific and practical purposes.

            One commonly used method for mushroom classification is morphological analysis, which involves examining the physical characteristics of the mushroom, such as its size, shape, and color. This method can be effective in many cases, but it can be challenging to apply consistently, especially when dealing with mushrooms that are similar in appearance.

            Another approach that has been explored is the use of chemical analysis to classify mushrooms. This method involves analyzing the chemical composition of the mushroom to identify unique characteristics that can be used to differentiate it from other species. This approach can be more reliable than morphological analysis, but it can be time-consuming and expensive.

            More recently, researchers have begun exploring the use of machine learning algorithms for mushroom classification. These algorithms can be trained on large datasets of mushroom images, allowing them to learn the characteristics that are important for distinguishing different species. This approach has the potential to be more accurate and efficient than traditional methods, but it also requires a significant amount of data and computational resources.

            Overall, it is clear that there are many different approaches to mushroom classification, each with its own strengths and limitations. Further research is needed to determine the most effective and practical methods for accurately classifying mushrooms in different contexts.


        
\end{document}
